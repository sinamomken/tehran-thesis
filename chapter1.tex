% !TeX root=main.tex
% دستور زیر باید در اولین فصل شما باشد. آن را حذف نکنید!
\pagenumbering{arabic}

\chapter{مقدمه}
\thispagestyle{empty}
\section{آشنایی با لاتک، زی‌پرشین و فایل‌های پایان‌نامه}
حروف‌چینی پروژه کارشناسی، پایان‌نامه یا رساله یکی از موارد پرکاربرد استفاده از
\lr{\LaTeX}
و زی‌پرشین
\cite{Khalighi87xepersian}
است. یک پروژه، پایان‌نامه یا رساله، احتیاج به تنظیمات زیادی از نظر صفحه‌آرایی دارد که وقت زیادی از دانشجو می‌گیرد. به دلیل قابلیت‌های بسیار لاتک در حروف‌چینی، یک کلاس با نام 
\lr{tehran-thesis}
برای حروف‌چینی پروژه‌ها، پایان‌نامه‌ها و رساله‌های دانشگاه تهران بر مبنای کلاس مشابه
\lr{IUST-Thesis}
آماده شده است. این کلاس و فایل‌های همراه آن به گونه‌ای طراحی شده است که مطابق با دستورالعمل نگارش و تدوین پایان‌نامه کارشناسی ارشد و دکتری پردیس دانشکده‌های فنی دانشگاه تهران
\cite{UTThesisGuide}
باشد.

دانشگاه تهران دو سند برای راهنمایی دانشجویان تهیه کرده است. در سند «دستورالعمل نگارش و تدوین پایان‌نامه» قالب و چگونگی صفحه‌آرایی پایان‌نامه، مانند اندازه و نوع قلم بخشهای مختلف، چینش فصلها، قالب مراجع و مواردی از این قبیل به دقت مشخص شده‌اند و در فایل «تمپلیت نگارش و تدوین پایان‌نامه» علاوه بر رعایت موارد نگارشی فوق، محتوای هر فصل پایان‌نامه شرح داده شده است. 
درصورت استفاده از این کلاس، دانشجو نیازی نیست که نگران مقوله اول باشد و لاتک همه کارها را برای وی انجام می‌دهد. فقط کافیست مطالب خود را تایپ و سند خود را با لاتک و ابزار آن اجرا کند تا پایان‌نامه خود را با قالب دانشگاه داشته باشد.

کلیه فایل‌های لازم برای حروف‌چینی با کلاس گفته شده، داخل پوشه‌ای به نام
\lr{tehran-thesis}
قرار داده شده است. توجه داشته باشید که برای استفاده از این کلاس باید فونت‌های
\lr{XB Niloofar}،
\lr{XB Zar}
و
\lr{XB Titre}
روی سیستم شما نصب باشند.
\section{این همه فایل؟!}\label{sec2}

از آنجایی که یک پایان‌نامه یا رساله، یک نوشته بلند محسوب می‌شود، لذا اگر همه تنظیمات و مطالب پایان‌نامه را داخل یک فایل قرار بدهیم، باعث شلوغی
و سردرگمی می‌شود. به همین خاطر، قسمت‌های مختلف پایان‌نامه یا رساله  داخل فایل‌های جداگانه قرار گرفته است. مثلاً تنظیمات پایه‌ای کلاس، داخل فایل
\lr{IUST-Thesis.cls}، 
تنظیمات قابل تغییر توسط کاربر، داخل 
\lr{commands.tex}،
قسمت مشخصات فارسی پایان‌نامه، داخل 
\lr{faTitle.tex}،
مطالب فصل اول، داخل 
\lr{intro}
و ... قرار داده شده است. نکته مهمی که در اینجا وجود دارد این است که از بین این  فایل‌ها، فقط فایل 
\lr{main.tex}
قابل اجرا است. یعنی بعد از تغییر فایل‌های دیگر، برای دیدن نتیجه تغییرات، باید این فایل را اجرا کرد. بقیه فایل‌ها به این فایل، کمک می‌کنند تا بتوانیم خروجی کار را ببینیم. اگر به فایل 
\lr{main.tex}
دقت کنید، متوجه می‌شوید که قسمت‌های مختلف پایان‌نامه، توسط دستورهایی مانند 
\lr{input}
و
\lr{include}
به فایل اصلی، یعنی 
\lr{main.tex}
معرفی شده‌اند. بنابراین، فایلی که همیشه با آن سروکار داریم، فایل 
\lr{main.tex}
است.
در این فایل، فرض شده است که پایان‌نامه یا رساله شما، از دو فصل و دو پیوست، تشکیل شده است. با این حال، خودتان می‌توانید به راحتی فصل‌ها و پیوست‌های بیشتر را به این مجموعه، اضافه کنید. این کار، بسیار ساده است. فرض کنید بخواهید یک فصل دیگر هم به پایان‌نامه، اضافه کنید. برای این کار، کافی است یک فایل با نام دلخواه مثلاً 
\lr{chapter3}
و با پسوند 
\lr{.tex}
بسازید و آن را داخل پوشه 
\lr{IUST-Thesis}
قرار دهید و سپس این فایل را با دستور 
\verb!% !TeX root=../main.tex
\chapter{روش تحقیق}
%\thispagestyle{empty} 
\section{مقدمه} 
این فصل، محل شرح کامل روش تحقیق است و بسته به نوع روش تحقیق و با نظر استاد راهنما می‌تواند «مواد و روش‌ها%
\LTRfootnote{Materials and Methods}»
نیز نام بگیرد. این فصل حدود ۱۵ صفحه است.

\section{محتوا (نام‌گذاری بر اساس روش تحقیق و مسأله مورد مطالعه)}
\subsection{علت انتخاب روش}
دلیل یا دلایل انتخاب روش تحقیق را تشریح می‌کند.

\subsection{تشریح کامل روش تحقیق}
برای اینکه پایان‌نامه دارای ارزش علمی باشد، باید قابل تکرار باشد و داوران و خوانندگان از امکان تکرارپذیر بودن کار شما مطمئن شوند. شما باید چگونگی تکرار آزمایش به وسیله دیگران را در این قسمت فراهم کنید. تکرارپذیری آزمایشات و روش شما، برابر با میزان پتانسیل تکرار نتایجِ برابر یا نزدیک به آن است. در زیر به تعدادی از روش‌های تحقیق اشاره شده است:
\begin{itemize}
	\item \textbf{روش تحقیق آزمایشگاهی}\\
	توصیف کامل برنامهٔ آزمایشگاهی شامل مواد مصرفی و نحوهٔ ساخت نمونه‌ها، شرح آزمایش‌ها شامل نحوه تنظیم و آماده‌سازی آزمایش‌ها و دستگاه‌های مورد استفاده، دقت و نحوهٔ کالیبره کردن، شرح دستگاه ساخته شده (در صورت ساخت) و ارائهٔ روش اعتبارسنجی.
	
	\item \textbf{روش تحقیق آماری}\\
	توصیف ابزارهای گردآوری اطلاعات کمی و کیفی، اندازهٔ نمونه‌ها، روش نمونه‌برداری، تشریح مبانی روش آماری و ارائهٔ روش اعتبارسنجی.
	
	\item \textbf{روش تحقیق نرم‌افزارنویسی}\\
	توصیف کامل برنامه‌نویسی، مبانی برنامه و ارائهٔ روش اعتبارسنجی.
	
	\item \textbf{روش تحقیق مطالعهٔ موردی}\\
	توصیف کامل محل و موضوع مطالعه، علت انتخاب مورد و پارامترهایی که تحت ارزیابی قرار داده می‌شوند و ارائهٔ روش اعتبارسنجی.

	\item \textbf{روش تحقیق تحلیلی یا مدل‌سازی}\\
	توصیف کامل مبانی یا اصول تحلیل یا مدل و ارائهٔ روش اعتبارسنجی آن. در ارائه مدل ریاضی معمولاً نیاز است اندیس‌ها، پارامترها، متغیرهای تصمیم و فرمول‌های مدل، به صورت سیستماتیک ارائه شوند. پیشنهاد می‌گردد برای نمایش اندیس‌ها، پارامترها و متغیرهای تصمیم از سه جدول به صورت زیر استفاده گردد:
	\begin{table}[ht]
		\caption{اندیس‌های به کار رفته در مدل ریاضی}
		\label{tab:modelIndices}
		\centering
		\onehalfspacing
		\begin{tabularx}{0.9\textwidth}{|r|X|}
			\hline
			$I, J$	& بیماران \\
			\hline
			$k$		& مرحله زمان‌بندی (بستری، اتاق عمل، ریکاوری) \\
			\hline
			$L_k$	& ماشین (تخت یا اتاق عمل) در مرحله $k$ \\
			\hline
			$n$		&  جراح \\
			\hline
		\end{tabularx}
	\end{table}
	
	\begin{table}[ht]
		\caption{پارامترهای مدل ریاضی}
		\label{tab:modelParameters}
		\centering
		\onehalfspacing
		\begin{tabularx}{0.9\textwidth}{|r|X|}
			\hline
			$t_{ik}$			& زمان خدمت‌دهی به بیمار در مرحله $k$ام \\
			\hline
			$\tilde{t}_{ik}$	& زمان فاری خدمت‌دهی به بیمار در محله $k$ام \\
			\hline
			$t_{ik}^p$			& مقدار بدبینانه (حداکثر) برای زمان خدمت‌دهی به بیمار در مرحله $k$ام \\
			\hline
			$t_{ik}^m$			& محتمل‌ترین مقدار برای زمان خدمت‌دهی به بیمار در مرحله $k$ام \\
			\hline
			$t_{ik}^o$			& مقدار خوشبینانه (حداقل) برای زمان خدمت‌دهی به بیمار در مرحله $k$ام \\
			\hline
		\end{tabularx}
	\end{table}
	
	\begin{table}[ht]
		\caption{متغیرهای مدل ریاضی}
		\label{tab:modelVariables}
		\centering
		\onehalfspacing
		\begin{tabularx}{0.9\textwidth}{|r|X|}
			\hline
			$X_{ild_{k}}$	& متغیر صفر-یک تخصیص بیمار به تخت/اتاق عمل\\
			\hline
			$S_{ild_{k}}$	& زمان شروع خدمت‌دهی به بیمار \\
			\hline
			$Y_{ijkl_{k}}$	& متغیر صفر-یک توالی بیماران \\
			\hline
			$V_{ni}$		& متغیر صفر-یک تخصیص جراح به بیمار‍‍ \\
			\hline
		\end{tabularx}
	\end{table}
	
	\item \textbf{روش تحقیق میدانی}\\
	چگونگی دستیابی به داده‌ها در میدان عمل و نحوه برداشت از پاسخ‌های دریافتی.
\end{itemize}!
داخل فایل
\lr{main.tex}
 قرار دهید.

\section{از کجا شروع کنم؟}
قبل از هر چیز، باید یک توزیع تِک مناسب مانند تک‌لایو
\lr{(TeXLive)}
را روی سیستم خود نصب کنید. تک‌لایو  را می‌توانید از 
 \href{http://www.tug.org/texlive}{سایت رسمی آن}%
\LTRfootnote{http://www.tug.org/texlive}
 دانلود کنید یا به صورت پستی از 
 \href{http://www.parsilatex.com}{سایت پارسی‌لاتک}%
\LTRfootnote{http://www.parsilatex.com}
سفارش دهید. مورد دوم حاوی مثالهای فارسی متنوعی شامل نمونه پایان‌نامه، نمونه مقاله، جدول و ... است که کارکردن اجزای مختلف آن مورد بررسی قرار گرفته است.

برای تایپ و پردازش اسناد لاتک باید از یک ویرایشگر مناسب استفاده کنید. به همراه تک‌لایو ویرایشگر \lr{TeXWroks} هست که می‌توانید از آن برای پردازش اسناد خود استفاده کنید. 
ویرایش‌گر 
\lr{Texmaker}
امکانات بیشتری دارد که نسخه بهینه شده آن برای زی‌پرشین با نام \lr{BiDi TeXMaker}  را می‌توانید  از 
 \href{http://www.parsilatex.com}{سایت پارسی‌لاتک} 
 دانلود کنید
 \footnote{توضیحات بیشتر درخصوص چگونگی اجرای اسناد زی‌پرشین را می‌توانید در فایل راهنمای دی‌وی‌دی پارسی‌لاتک ببینید.}.
در مرحله بعد، سعی کنید که  یک پشتیبان از پوشه 
\lr{IUST-Thesis}
 بگیرید و آن را در یک جایی از هارددیسک سیستم خود ذخیره کنید تا در صورت خراب کردن فایل‌هایی که در حال حاضر، با آن‌ها کار می‌کنید، همه چیز را از 
 دست ندهید.
 
 حال اگر نوشتن \پ اولین تجربه شما از کار با لاتک است، توصیه می‌شود که یک‌بار به صورت اجمالی، کتاب «%
\href{http://www.tug.ctan.org/tex-archive/info/lshort/persian/lshort.pdf}{مقدمه‌ای نه چندان کوتاه بر
\lr{\LaTeXe}}\footnote{اگر تک‌لایو کامل را داشته باشید، این کتاب را هم دارید. در هر صورت از آدرس زیر قابل دانلود است:\\
\lr{\url{http://www.tug.ctan.org/tex-archive/info/lshort/persian/lshort.pdf}\hfill}}»
   ترجمه دکتر مهدی امیدعلی را مطالعه کنید. این کتاب، کتاب بسیار کاملی است که خیلی از نیازهای شما در ارتباط با حروف‌چینی را برطرف می‌کند.
اگر عجله دارید، برخی دستورات پایه‌ای مورد نیاز در فصل \ref{Chap:latexIntro} بیان شده‌اند.
 
 
بعد از موارد گفته شده، فایل 
\lr{main.tex}
و
\lr{faTitle}
را باز کنید و مشخصات پایان‌نامه خود مثل نام، نام خانوادگی، عنوان پایان‌نامه و ... را جایگزین مشخصات موجود در فایل
\lr{faTitle}
 کنید. دقت داشته باشید که نیازی نیست 
نگران چینش این مشخصات در فایل پی‌دی‌اف خروجی باشید. فایل 
\lr{IUST-Thesis.cls}
همه این کارها را به طور خودکار برای شما انجام می‌دهد. در ضمن، موقع تغییر دادن دستورهای داخل فایل
\lr{faTitle}
 کاملاً دقت کنید. این دستورها، خیلی حساس هستند و ممکن است با یک تغییر کوچک، موقع اجرا، خطا بگیرید. برای دیدن خروجی کار، فایل 
\lr{faTitle}
 را 
\lr{Save}، 
(نه 
\lr{Save As})
کنید و بعد به فایل 
\lr{main.tex}
برگشته و آن را اجرا کنید
\footnote{فایلهای این مجموعه به گونه‌ای هستند که در \lr{TeXWorks}  بدون برگشتن به فایل اصلی، می‌توانید سند خود را اجرا کنید. }.
 حال اگر می‌خواهید مشخصات انگلیسی \پ را هم عوض کنید، فایل 
\lr{enTitle}
را باز کنید و مشخصات داخل آن را تغییر دهید.%
%\RTLfootnote{
%برای نوشتن پروژه کارشناسی، نیازی به وارد کردن مشخصات انگلیسی پروژه نیست. بنابراین، این مشخصات، به طور خودکار،
%نادیده گرفته می‌شود.
%}
 در اینجا هم برای دیدن خروجی، باید این فایل را 
\lr{Save}
کرده و بعد به فایل 
\lr{main.tex}
برگشته و آن را اجرا کرد.

برای راحتی بیشتر، 
فایل 
\lr{IUST-Thesis.cls}
طوری طراحی شده است که کافی است فقط  یک‌بار مشخصات \پ  را وارد کنید. هر جای دیگر که لازم به درج این مشخصات باشد، این مشخصات به طور خودکار درج می‌شود. با این حال، اگر مایل بودید، می‌توانید تنظیمات موجود را تغییر دهید. توجه داشته باشید که اگر کاربر مبتدی هستید و یا با ساختار فایل‌های  
\lr{cls}
 آشنایی ندارید، به هیچ وجه به این فایل، یعنی فایل 
\lr{IUST-Thesis.cls}
دست نزنید.

نکته دیگری که باید به آن توجه کنید این است که در فایل 
\lr{IUST-Thesis.cls}،
سه گزینه به نام‌های
\lr{bsc}،
\lr{msc}
و
\lr{phd}
برای تایپ پروژه، پایان‌نامه و رساله،
طراحی شده است. بنابراین اگر قصد تایپ پروژه کارشناسی، پایان‌نامه یا رساله را دارید، 
 در فایل 
\lr{main.tex}
باید به ترتیب از گزینه‌های
\lr{bsc}،
\lr{msc}
و
\lr{phd}
استفاده کنید. با انتخاب هر کدام از این گزینه‌ها، تنظیمات مربوط به آنها به طور خودکار، اعمل می‌شود.    
فقط اطلاعات صفحه مربوط با تاییدیه هیات داوران باید به صورت دستی وارد شوند.


\section[مطالب پروژه را چطور بنویسم؟]
{مطالب \پ را چطور بنویسم؟}
\subsection{نوشتن فصل‌ها}
همان‌طور که در بخش \ref{sec2} گفته شد، برای جلوگیری از شلوغی و سردرگمی کاربر در هنگام حروف‌چینی، قسمت‌های مختلف \پ از جمله فصل‌ها، در فایل‌های جداگانه‌ای قرار داده شده‌اند. 
بنابراین، اگر می‌خواهید مثلاً مطالب فصل ۱ را تایپ کنید، باید فایل‌های 
\lr{main.tex}
و
\lr{intro}
را باز کنید و مطالب خود را جایگزین محتویات داخل فایل 
\lr{intro}
نمایید. دقت داشته باشید که در ابتدای برخی فایلها دستوراتی نوشته شده است و از شما خواسته شده است که آن دستورات را حذف نکنید.

%توجه کنید که همان‌طور که قبلاً هم گفته شد، تنها فایل قابل اجرا، فایل 
%\lr{main.tex}
%است. لذا برای دیدن حاصل (خروجی) فایل خود، باید فایل  
%\lr{intro}
%را 
%\lr{Save}
%کرده و سپس فایل 
%\lr{main.tex}
%را اجرا کنید. یک نکته بدیهی که در اینجا وجود دارد، این است که لازم نیست که فصل‌های \پ را به ترتیب تایپ کنید. می‌توانید ابتدا مطالب فصل ۳ را تایپ کنید و سپس مطالب فصل ۱ را تایپ کنید. 

نکته بسیار مهمی که در اینجا باید گفته شود این است که سیستم \lr{\TeX}، محتویات یک فایل تِک را به ترتیب پردازش می‌کند.  بنابراین، اگر مثلاً  دو فصل اول خود را نوشته و خروجی آنها را دیده‌اید و مشغول تایپ مطالب فصل ۳ هستید، بهتر است
که دو دستور 
\verb!\include{intro}!
و
\verb!\include{latexIntro}!
را در فایل 
\lr{main.tex}،
غیرفعال%
\footnote{
برای غیرفعال کردن یک دستور، کافی است در ابتدای آن، یک علامت
\%
 بگذارید.
}
 کنید.  در غیر این صورت، ابتدا مطالب دو فصل اول  پردازش شده و سپس مطالب فصل ۳ پردازش می‌شود و این کار باعث طولانی شدن زمان اجرا می‌شود. هر زمان که خروجی کل \پ خود را خواستید تمام فصلها را از حالت توضیح خارج کنید.

\subsection{مراجع}
برای وارد کردن مراجع \پ خود، کافی است فایل 
\lr{MyReferences.bib}
را باز کرده و مراجع خود را مانند مراجع داخل آن، وارد کنید.  سپس از \lr{bibtex} برای تولید مراجع با قالب مناسب استفاده کنید. برای توضیحات بیشتر بخش \ref{Sec:Ref} و پیوست \ref{App:RefMan} را ببینید.


\subsection{واژه‌نامه فارسی به انگلیسی و برعکس}
برای وارد کردن واژه‌نامه فارسی به انگلیسی و برعکس، چنانچه کاربر مبتدی هستید، بهتر است مانند روش بکار رفته در فایل‌های 
\lr{dicfa2en}
و
\lr{dicen2fa}
عمل کنید. اما چنانچه کاربر پیشرفته هستید، بهتر است از بسته
\lr{glossaries}
استفاده کنید. راهنمای این بسته را می‌توانید به راحتی و با یک جستجوی ساده در اینترنت پیدا کنید.
\subsection{نمایه}
برای وارد کردن نمایه، باید از 
\lr{xindy}
استفاده کنید. 
%زیرا 
%\lr{MakeIndex}
%با حروف «گ»، «چ»، «پ»، «ژ» و «ک» مشکل دارد و ترتیب الفبایی این حروف را رعایت نمی‌کند. همچنین، فاصله بین هر گروه از کلمات در 
%\lr{MakeIndex}،
%به درستی رعایت نمی‌شود که باعث زشت شدن حروف‌چینی این قسمت می‌شود. 
راهنمای چگونگی کار با 
\lr{xindy} 
را می‌توانید در تالار گفتگوی پارسی‌لاتک و یا مثالهای موجود در مجموعه پارسی‌لاتک، پیدا کنید.

\section{اگر سوالی داشتم، از کی بپرسم؟}
برای پرسیدن سوال‌های خود موقع حروف‌چینی با زی‌پرشین،  می‌توانید به
 \href{http://forum.parsilatex.com}{تالار گفتگوی پارسی‌لاتک}%
\LTRfootnote{http://forum.parsilatex.com}
مراجعه کنید. شما هم می‌توانید روزی به سوال‌های دیگران در این تالار، جواب بدهید.
    
\section{جمع‌بندی}
بسته‌ی زی‌پرشین و بسیاری بسته‌های مرتبط با آن مانند \lr{bidi} و \lr{Persian-bib}، مجموعه پارسی‌لاتک، مثالهای مختلف موجود در آن، استیلهای مختلف پایان‌نامه دانشگاههای مختلف، سایت پارسی‌لاتک همه به صورت داوطلبانه توسط افراد گروه پارسی‌لاتک و بدون هیچ کمک مالی انجام شده‌اند. کار اصلی نوشتن و توسعه زی‌پرشین توسط آقای وفا خلیقی انجام شده است که این کار بزرگ را به انجام رساندند.
اگر مایل به کمک مالی به گروه پارسی‌لاتک هستید کمک‌های مالی خود را به  شماره حساب 
زیر نزد بانک ملی، به نام هادی صفی‌اقدم واریز نمایید:
\begin{center}
شماره حساب: ۰۱۰۱۲۰۰۰۷۰۰۰۳

شماره کارت: 
\lr{6037-9910-4168-7363}

شماره شبا: 
\lr{IR72-0170-0000-0010-1200-0700-03}
\end{center}
لطفاً پس از واریز وجه، موضوع را از طریق ایمیل به آقای صفی‌اقدم اطلاع دهید (\lr{hadi.safiaghdam@gmail.com}).