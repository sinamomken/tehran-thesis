% !TEX TS-program = XeLaTeX
% Command(s) for running this example:
%    latexmk -e "$pdflatex=q/xelatex -synctex=1 -interaction=nonstopmode/" -pdf main.tex
% OR
% 	 xelatex main
% 	 bibtex8 -W -c cp1256fa main
%      xindy -L persian -C utf8 -M texindy main
% 	 xelatex main
% 	 xelatex main
% End of Commands

%        نمونه پایان‌نامه آماده شده با استفاده از کلاس IUST-Thesis، نگارش 0.6
% 		محمود امین‌طوسی، دانشگاه تربیت معلم سبزوار، http://profsite.sttu.ac.ir/mamintoosi/
% 		گروه پارسی‌لاتک  http://www.parsilatex.com
%        این نسخه، بر اساس نسخه‌ 0.4 از کلاس Tabriz_Thesis آقای وحید دامن‌افشان آماده شده است. http://damanafshan.tk
%        
%        تغییرات:
%        نسخه 0.6:
%        اصلاح مشکل بسته subfig 
%----------------------------------------------------------------------------------------------
%        اگر قصد نوشتن پروژه کارشناسی را دارید، در خط زیر به جای msc، کلمه bsc و اگر قصد نوشتن پروژه دکترا
%        را دارید، کلمه phd را قرار دهید. کلیه تنظیمات لازم، به طور خودکار، اعمال می‌شود.

%        اگر مایلید پایان‌نامه شما دورو باشد به جای oneside  در خط زیر از twoside استفاده کنید
\documentclass[oneside,openany,msc]{IUST-Thesis}

% مشخصات پایان‌نامه را در فایلهای faTitle و enTitle وارد نمایید.

%       فایل commands.tex را مطالعه کنید؛ چون دستورات مربوط به فراخوانی بسته زی‌پرشین 
%       و دیگر بسته‌ها و ... در این فایل قرار دارد و بهتر است که با نحوه استفاده از آنها آشنا شوید.
% در این فایل، دستورها و تنظیمات مورد نیاز، آورده شده است.
%-------------------------------------------------------------------------------------------------------------------

% در ورژن جدید زی‌پرشین برای تایپ متن‌های ریاضی، این سه بسته، حتماً باید فراخوانی شود
\usepackage{amsthm,amssymb,amsmath}
% بسته‌ای برای تنطیم حاشیه‌های بالا، پایین، چپ و راست صفحه
\usepackage[top=40mm, bottom=40mm, left=25mm, right=35mm]{geometry}
% بسته‌‌ای برای ظاهر شدن شکل‌ها و تصاویر متن
\usepackage{graphicx}
% بسته‌ای برای رسم کادر
\usepackage{framed} 
% بسته‌‌ای برای چاپ شدن خودکار تعداد صفحات در صفحه «معرفی پایان‌نامه»
\usepackage{lastpage}
% بسته‌ و دستوراتی برای ایجاد لینک‌های رنگی با امکان جهش
\usepackage[pagebackref=false,colorlinks,linkcolor=blue,citecolor=blue]{hyperref}
% چنانچه قصد پرینت گرفتن نوشته خود را دارید، خط بالا را غیرفعال و  از دستور زیر استفاده کنید چون در صورت استفاده از دستور زیر‌‌، 
% لینک‌ها به رنگ سیاه ظاهر خواهند شد که برای پرینت گرفتن، مناسب‌تر است
%\usepackage[pagebackref=false]{hyperref}
% بسته‌ لازم برای تنظیم سربرگ‌ها
\usepackage{fancyhdr}
%
\usepackage{setspace}
\usepackage{algorithm}
\usepackage{algorithmic}
\usepackage{subfigure}
\usepackage[subfigure]{tocloft}


% بسته‌ای برای ظاهر شدن «مراجع» و «نمایه» در فهرست مطالب
\usepackage[nottoc]{tocbibind}
% دستورات مربوط به ایجاد نمایه
\usepackage{makeidx}
\makeindex
%%%%%%%%%%%%%%%%%%%%%%%%%%
% فراخوانی بسته زی‌پرشین و تعریف قلم فارسی و انگلیسی
\usepackage{xepersian}
\settextfont[Scale=1]{XB Niloofar}
\setlatintextfont[Scale=0.9]{Liberation Serif}

%%%%%%%%%%%%%%%%%%%%%%%%%%
% چنانچه می‌خواهید اعداد در فرمول‌ها، انگلیسی باشد، خط زیر را غیرفعال کنید
\setdigitfont[Scale=1]{XB Zar}%{Persian Modern}
%%%%%%%%%%%%%%%%%%%%%%%%%%
% تعریف قلم‌های فارسی و انگلیسی اضافی برای استفاده در بعضی از قسمت‌های متن
\defpersianfont\titlefont[Scale=1]{XB Titre}
% \defpersianfont\iranic[Scale=1.1]{XB Zar Oblique}%Italic}%
% \defpersianfont\nastaliq[Scale=1.2]{IranNastaliq}

%%%%%%%%%%%%%%%%%%%%%%%%%%
% دستوری برای حذف کلمه «چکیده»
\renewcommand{\abstractname}{}
% دستوری برای حذف کلمه «abstract»
%\renewcommand{\latinabstract}{}
% دستوری برای تغییر نام کلمه «اثبات» به «برهان»
\renewcommand\proofname{\textbf{برهان}}
% دستوری برای تغییر نام کلمه «کتاب‌نامه» به «مراجع»
\renewcommand{\bibname}{مراجع}
% دستوری برای تعریف واژه‌نامه انگلیسی به فارسی
\newcommand\persiangloss[2]{#1\dotfill\lr{#2}\\}
% دستوری برای تعریف واژه‌نامه فارسی به انگلیسی 
\newcommand\englishgloss[2]{#2\dotfill\lr{#1}\\}
% تعریف دستور جدید «\پ» برای خلاصه‌نویسی جهت نوشتن عبارت «پروژه/پایان‌نامه/رساله»
\newcommand{\پ}{پروژه/پایان‌نامه/رساله }

%\newcommand\BackSlash{\char`\\}

%%%%%%%%%%%%%%%%%%%%%%%%%%
\SepMark{-}

% تعریف و نحوه ظاهر شدن عنوان قضیه‌ها، تعریف‌ها، مثال‌ها و ...
\theoremstyle{definition}
\newtheorem{definition}{تعریف}[section]
\theoremstyle{theorem}
\newtheorem{theorem}[definition]{قضیه}
\newtheorem{lemma}[definition]{لم}
\newtheorem{proposition}[definition]{گزاره}
\newtheorem{corollary}[definition]{نتیجه}
\newtheorem{remark}[definition]{ملاحظه}
\theoremstyle{definition}
\newtheorem{example}[definition]{مثال}

%\renewcommand{\theequation}{\thechapter-\arabic{equation}}
%\def\bibname{مراجع}
\numberwithin{algorithm}{chapter}
\def\listalgorithmname{فهرست الگوریتم‌ها}
\def\listfigurename{فهرست تصاویر}
\def\listtablename{فهرست جداول}

%%%%%%%%%%%%%%%%%%%%%%%%%%%%
% دستورهایی برای سفارشی کردن سربرگ صفحات
% \newcommand{\SetHeader}{
% \csname@twosidetrue\endcsname
% \pagestyle{fancy}
% \fancyhf{} 
% \fancyhead[OL,EL]{\thepage}
% \fancyhead[OR]{\small\rightmark}
% \fancyhead[ER]{\small\leftmark}
% \renewcommand{\chaptermark}[1]{%
% \markboth{\thechapter-\ #1}{}}
% }
%%%%%%%%%%%%5
%\def\MATtextbaseline{1.5}
%\renewcommand{\baselinestretch}{\MATtextbaseline}
\doublespacing
%%%%%%%%%%%%%%%%%%%%%%%%%%%%%
% دستوراتی برای اضافه کردن کلمه «فصل» در فهرست مطالب

\newlength\mylenprt
\newlength\mylenchp
\newlength\mylenapp

\renewcommand\cftpartpresnum{\partname~}
\renewcommand\cftchappresnum{\chaptername~}
\renewcommand\cftchapaftersnum{:}

\settowidth\mylenprt{\cftpartfont\cftpartpresnum\cftpartaftersnum}
\settowidth\mylenchp{\cftchapfont\cftchappresnum\cftchapaftersnum}
\settowidth\mylenapp{\cftchapfont\appendixname~\cftchapaftersnum}
\addtolength\mylenprt{\cftpartnumwidth}
\addtolength\mylenchp{\cftchapnumwidth}
\addtolength\mylenapp{\cftchapnumwidth}

\setlength\cftpartnumwidth{\mylenprt}
\setlength\cftchapnumwidth{\mylenchp}	

\makeatletter
{\def\thebibliography#1{\chapter*{\refname\@mkboth
   {\uppercase{\refname}}{\uppercase{\refname}}}\list
   {[\arabic{enumi}]}{\settowidth\labelwidth{[#1]}
   \rightmargin\labelwidth
   \advance\rightmargin\labelsep
   \advance\rightmargin\bibindent
   \itemindent -\bibindent

   \listparindent \itemindent
   \parsep \z@
   \usecounter{enumi}}
   \def\newblock{}
   \sloppy
   \sfcode`\.=1000\relax}}
\makeatother


\begin{document}

\pagenumbering{harfi}
% !TeX root=main.tex
% در این فایل، عنوان پایان‌نامه، مشخصات خود، متن تقدیمی‌، ستایش، سپاس‌گزاری و چکیده پایان‌نامه را به فارسی، وارد کنید.
% توجه داشته باشید که جدول حاوی مشخصات پروژه/پایان‌نامه/رساله و همچنین، مشخصات داخل آن، به طور خودکار، درج می‌شود.
%%%%%%%%%%%%%%%%%%%%%%%%%%%%%%%%%%%%
% دانشگاه خود را وارد کنید
\university{دانشگاه تهران}
% دانشکده، آموزشکده و یا پژوهشکده  خود را وارد کنید
\faculty{دانشکده علوم مهندسی}
% گروه آموزشی خود را وارد کنید
\department{گروه الگوریتم‌ها و محاسبات}
% گروه آموزشی خود را وارد کنید
\subject{مهندسی کامپیوتر}
% گرایش خود را وارد کنید
\field{الگوریتم‌ها و محاسبات}
% عنوان پایان‌نامه را وارد کنید
\title{نوشتن پروژه، پایان‌نامه و رساله با استفاده از کلاس 
tehran-thesis}
% نام استاد(ان) راهنما را وارد کنید
\firstsupervisor{استاد راهنمای اول}
\secondsupervisor{استاد راهنمای دوم}
% نام استاد(دان) مشاور را وارد کنید. چنانچه استاد مشاور ندارید، دستور پایین را غیرفعال کنید.
\firstadvisor{استاد مشاور اول}
%\secondadvisor{استاد مشاور دوم}
% نام دانشجو را وارد کنید
\name{سینا}
% نام خانوادگی دانشجو را وارد کنید
\surname{ممکن}
% شماره دانشجویی دانشجو را وارد کنید
\studentID{810893024}
% تاریخ پایان‌نامه را وارد کنید
\thesisdate{مرداد ۱۳۹۶}
% به صورت پیش‌فرض برای پایان‌نامه‌های کارشناسی تا دکترا به ترتیب از عبارات «پروژه»، «پایان‌نامه» و «رساله» استفاده می‌شود؛ اگر  نمی‌پسندید هر عنوانی را که مایلید در دستور زیر قرار داده و آنرا از حالت توضیح خارج کنید.
%\projectLabel{پایان‌نامه}

% به صورت پیش‌فرض برای عناوین مقاطع تحصیلی کارشناسی تا دکترا به ترتیب از عبارت «کارشناسی»، «کارشناسی ارشد» و «دکتری» استفاده می‌شود؛ اگر نمی‌پسندید هر عنوانی را که مایلید در دستور زیر قرار داده و آنرا از حالت توضیح خارج کنید.
%\degree{}

\coverPage
\besmPage
\titlePage
\davaranPage

%\vspace{.5cm}
%%%%%%%%%%%%%%%%%%%%%%%%%%%%%%%%%%%%%%%%%%%%%%%%%%%%%%
% در این قسمت اسامی اساتید راهنما، مشاور و داور باید به صورت دستی وارد شوند
%\renewcommand{\arraystretch}{1.2}
\begin{center}
\begin{tabular}{| p{8mm} | p{18mm} | p{.17\textwidth} |p{14mm}|p{.2\textwidth}|c|}
\hline
ردیف & مشخصات هیئت داوران & نام و نام خانوادگی & مرتبه \newline دانشگاهی &	دانشگاه یا مؤسسه & امضـــــــــــــا \\
\hline
۱ &	استاد راهنما & دکتر دارا معظمی & استاد & دانشگاه تهران & \\

۲ &	استاد راهنما & دکتر کاوه کاووسی & استادیار & دانشگاه تهران & \\
\hline
۳ & استاد مشاور & دکتر علی‌محمد بنائی‌مقدم & استادیار & دانشگاه تهران & \\
\hline
۴ & استاد داور داخلی & دکتر داور داخلی & دانشیار & دانشگاه تهران & \\
\hline
۵ & استاد مدعو & دکتر داور خارجی & دانشیار & دانشگاه داور خارجی & \\
\hline
۶ & نماینده تحصیلات تکمیلی دانشکده & دکتر نماینده & دانشیار & دانشگاه تهران & \\
\hline
\end{tabular}
\end{center}

\esalatPage
\mojavezPage

%%%%%%%%%%%%%%%%%%%%%%%%%%%%%%%%%%%%%%%%%%%%%%%%%%%%%%
% چنانچه مایل به چاپ صفحات «تقدیم»، «نیایش» و «سپاس‌گزاری» در خروجی نیستید، خط‌های زیر را با گذاشتن ٪  در ابتدای آنها غیرفعال کنید.

%% پایان‌نامه خود را تقدیم کنید! %%
 \newpage
\thispagestyle{empty}
{\Large تقدیم به:}\\
\begin{flushleft}
{\huge
همسر و فرزندانم\\
\vspace{7mm}
و\\
\vspace{7mm}
پدر و مادرم
}
\end{flushleft}


%% سپاس‌گزاری %%
\begin{acknowledgementpage}
سپاس خداوندگار حکیم را که با لطف بی‌کران خود، آدمی را زیور عقل آراست.


در آغاز وظیفه‌  خود  می‌دانم از زحمات بی‌دریغ استاد  راهنمای خود،  جناب آقای دکتر ...، صمیمانه تشکر و  قدردانی کنم  که قطعاً بدون راهنمایی‌های ارزنده‌  ایشان، این مجموعه  به انجام  نمی‌رسید.

از جناب  آقای  دکتر ...   که زحمت  مطالعه و مشاوره‌  این رساله را تقبل  فرمودند و در آماده سازی  این رساله، به نحو احسن اینجانب را مورد راهنمایی قرار دادند، کمال امتنان را دارم.

همچنین لازم می‌دانم از پدید آورندگان بسته زی‌پرشین، مخصوصاً جناب آقای  وفا خلیقی، که این پایان‌نامه با استفاده از این بسته، آماده شده است و همه دوستانمان در گروه پارسی‌لاتک کمال قدردانی را داشته باشم.

 در پایان، بوسه می‌زنم بر دستان خداوندگاران مهر و مهربانی، پدر و مادر عزیزم و بعد از خدا، ستایش می‌کنم وجود مقدس‌شان را و تشکر می‌کنم از خانواده عزیزم به پاس عاطفه سرشار و گرمای امیدبخش وجودشان، که بهترین پشتیبان من بودند.
% با استفاده از دستور زیر، امضای شما، به طور خودکار، درج می‌شود.
\signature 
\end{acknowledgementpage}
%%%%%%%%%%%%%%%%%%%%%%%%%%%%%%%%%%%%
% کلمات کلیدی پایان‌نامه را وارد کنید
\keywords{زی‌پرشین، لاتک، قالب پایان‌نامه، الگو}
%چکیده پایان‌نامه را وارد کنید، برای ایجاد پاراگراف جدید از \\ استفاده کنید. اگر خط خالی دشته باشید، خطا خواهید گرفت.
\fa-abstract{
این پایان‌نامه، به بحث در مورد نوشتن پروژه، پایان‌نامه و رساله با استفاده از کلاس 
\lr{tehran-thesis}
می‌پردازد.
حروف‌چینی پروژه کارشناسی، پایان‌نامه یا رساله یکی از موارد پرکاربرد استفاده از زی‌پرشین است. 
زی‌پرشین بسته‌ای است که به همت آقای وفا خلیقی آماده شده است و امکان حروف‌چینی فارسی در \lr{\LaTeXe}{} را  برای فارسی‌زبانان فراهم کرده است.
از جمله مزایای لاتک آن است که در صورت وجود یک کلاس آماده برای حروف‌چینی یک سند خاص مانند یک پایان‌نامه، کاربر بدون درگیری با جزییات حروف‌چینی و صفحه‌آرایی می‌تواند سند خود را آماده نماید.
\\
شاید با قالب‌های لاتکی که برخی از مجلات برای مقالات خود عرضه می‌کنند مواجه شده باشید. اگر نظیر این کار در دانشگاههای مختلف برای اسناد متنوع آنها مانند پایا‌ن‌نامه‌ها آماده شود، دانشجویان به جای وقت گذاشتن روی صفحه‌آرایی مطالب خود، روی محتوای متن خود تمرکز خواهند نمود. به علاوه با آشنایی با لاتک خواهند توانست از امکانات بسیار این نرم‌افزار جهت نمایش بهتر دست‌آوردهای خود استفاده کنند.
به همین خاطر، یک کلاس با نام 
\lr{tehran-thesis}
 برای حروف‌چینی پروژه‌ها، پایان‌نامه‌ها و رساله‌های دانشگاه علم و صنعت ایران با استفاده از نرم‌افزار زی‌پرشین،  آماده شده است. این فایل به 
گونه‌ای طراحی شده است که کلیات خواسته‌های مورد نیاز  مدیریت تحصیلات تکمیلی دانشگاه علم و صنعت ایران را برآورده می‌کند و نیز، حروف‌چینی بسیاری از قسمت‌های آن، به طور خودکار انجام می‌شود.
}

%\fa-abstract{
%%این پایان‌نامه، به بحث در مورد نوشتن پروژه، پایان‌نامه و رساله با استفاده از کلاس 
%%\lr{IUST-Thesis}
%%می‌پردازد. 
%%حروف‌چینی پروژه کارشناسی، پایان‌نامه یا رساله یکی از موارد پرکاربرد استفاده از زی‌پرشین است. 
%%زی‌پرشین بسته‌ای است که به همت آقای وفا خلیقی آماده شده است و امکان حروف‌چینی فارسی در  را  برای فارسی‌زبانان فراهم کرده است.
%%از جمله مزایای لاتک آن است که در صورت وجود یک کلاس آماده برای حروف‌چینی یک سند خاص مانند یک پایان‌نامه، کاربر بدون درگیری با جزییات حروف‌چینی و صفحه‌آرایی می‌توان سند خود را آماده نماید.
%
%شاید با قالب‌های لاتکی که برخی از مجلات برای مقالات خود عرضه می‌کنند مواجه شده باشید. اگر نظیر این کار در دانشگاههای مختلف برای اسناد متنوع آنها مانند پایا‌ن‌نامه‌ها آماده شود، دانشجویان به جای وقت گذاشتن روی صفحه‌آرایی مطالب خود، روی محتوای متن خود تمرکز خواهند نمود. به علاوه با آشنایی با لاتک خواهند توانست از امکانات بسیار این نرم‌افزار جهت نمایش بهتر دستآوردهای خود استفاده کنند.
%به همین خاطر، یک کلاس با نام 
%\lr{IUST-Thesis}
% برای حروف‌چینی پروژه‌ها، پایان‌نامه‌ها و رساله‌های دانشگاه علم و صنعت ایران با استفاده از نرم‌افزار زی‌پرشین،  آماده شده است. این فایل به 
%گونه‌ای طراحی شده است که کلیات خواسته‌های مورد نیاز  مدیریت تحصیلات تکمیلی دانشگاه علم و صنعت ایران را برآورده می‌کند و نیز، حروف‌چینی بسیاری از قسمت‌های آن، به طور خودکار انجام می‌شود.
%}

\abstractPage

\newpage\clearpage
\tableofcontents

\newpage
\listoffigures \newpage
\listoftables  \newpage
\addcontentsline{toc}{chapter}{\listalgorithmname}
\listofalgorithms \newpage
\newacronym{a}{$a$}{\rl{شتاب ($m/s^2$)}}
\newacronym{F}{$F$}{\rl{نیرو ($N$)}}


\pagestyle{fancy}
% اگر شما فصل اول  خود را در فایلی به جز chapter1 همراه با این کلاس نوشته‌اید باید چندخط اول chapter1 را در فایل خود کپی کنید.
\include{intro}			% فصل اول: مقدمه
\include{latexIntro}		% فصل دوم: آشنایی مقدماتی با لاتک

% مراجع
\pagestyle{empty}
{
\onehalfspacing
\bibliographystyle{acm-fa}%{chicago-fa}%{plainnat-fa}%
\bibliography{MyReferences}
}

\pagestyle{fancy}

\appendix                           %فصلهای پس از این قسمت به عنوان ضمیمه خواهند آمد.
% اگر شما پیوست اول  خود را در فایلی به جز appendix1 همراه با این کلاس نوشته‌اید باید چندخط اول appendix1 را در فایل خود کپی کنید.
% !TeX root=main.tex
% دستورات زیر باید در اولین فایل پیوست باشند. آنها را حذف نکنید!
\addtocontents{toc}{
    \protect\renewcommand\protect\cftchappresnum{\appendixname~}%
    \protect\setlength{\cftchapnumwidth}{\mylenapp}}%
    
\chapter{مدیریت مراجع در لاتک}\label{App:RefMan}
\thispagestyle{empty}

در بخش \ref{Sec:Ref} اشاره شد که با دستور 
 \lr{\textbackslash bibitem}
  می‌توان یک مرجع را تعریف نمود و با فرمان
 \lr{\textbackslash cite}
  به آن ارجاع داد. این روش برای تعداد مراجع زیاد و تغییرات آنها مناسب نیست. در ادامه به صورت مختصر توضیحی در خصوص برنامه \lr{BibTeX} که همراه با توزیع‌های معروف تِک عرضه می‌شود و نحوه استفاده از آن در زی‌پرشین خواهیم داشت.

\section{ مدیریت مراجع با  \texorpdfstring{\lr{Bib\TeX}}{Bib\TeX} }
یکی از روش‌های قدرتمند و انعطاف‌پذیر برای نوشتن مراجع مقالات و مدیریت مراجع در لاتک، استفاده از  \lr{BibTeX} است.
 روش کار با  \lr{BibTeX} به این صورت است که مجموعه‌ی همه‌ی مراجعی را که در \پ استفاده کرده یا خواهیم کرد، 
در پرونده‌ی جداگانه‌ای نوشته و به آن فایل در سند خودمان به صورت مناسب لینک می‌دهیم.
 کنفرانس‌ها یا مجله‌های گوناگون برای نوشتن مراجع، قالب‌ها یا قراردادهای متفاوتی دارند که به آنها استیلهای مراجع گفته می‌شود.
 در این حالت به کمک ‌استیل‌های \lr{BibTeX} خواهید توانست تنها با تغییر یک پارامتر در پرونده‌ی ورودی خود، مراجع را مطابق قالب موردنظر تنظیم کنید. 
 بیشتر مجلات و کنفرانس‌های معتبر یک پرونده‌ی سبک (\lr{BibTeX Style}) با پسوند \lr{bst} در وب‌گاه خود می‌گذارند که برای همین منظور طراحی شده است.

به جز نوشتن مقالات این سبک‌ها کمک بسیار خوبی برای تهیه‌ی مستندات علمی همچون پایان‌نامه‌هاست که فرد می‌تواند هر قسمت از کارش را که نوشت مراجع مربوطه را به بانک مراجع خود اضافه نماید. با داشتن چنین بانکی از مراجع، وی خواهد توانست به راحتی یک یا چند ارجاع به مراجع و یا یک یا چند بخش را حذف یا اضافه ‌نماید؛ 
مراجع به صورت خودکار مرتب شده و فقط مراجع ارجاع داده شده در قسمت کتاب‌نامه خواهندآمد. قالب مراجع به صورت یکدست مطابق سبک داده شده بوده و نیازی نیست که کاربر درگیر قالب‌دهی به مراجع باشد. 
در این جا مجموعه‌ سبک‌های بسته \lr{Persian-bib} که برای  زی‌پرشین آماده شده‌اند به صورت مختصر معرفی شده و روش کار با آن‌ها گفته می‌شود. برای اطلاع بیشتر به راهنمای بسته‌ی \lr{Persian-bib} مراجعه فرمایید.
\subsection{سبک‌های فعلی قابل استفاده در زی‌پرشین}
در حال حاضر فایلهای سبک زیر برای استفاده در زی‌پرشین آماده شده‌اند:

\singlespacing
\begin{description}
\item [unsrt-fa.bst] این سبک متناظر با \lr{unsrt.bst} می‌باشد. مراجع به ترتیب ارجاع در متن ظاهر می‌شوند.
\item [plain-fa.bst] این سبک متناظر با \lr{plain.bst} می‌باشد. مراجع بر اساس نام‌خانوادگی نویسندگان، به ترتیب صعودی مرتب می‌شوند.
 همچنین ابتدا مراجع فارسی و سپس مراجع انگلیسی خواهند آمد.
\item [acm-fa.bst] این سبک متناظر با \lr{acm.bst} می‌باشد. شبیه \lr{plain-fa.bst} است.  قالب مراجع کمی متفاوت است. اسامی نویسندگان انگلیسی با حروف بزرگ انگلیسی نمایش داده می‌شوند. (مراجع مرتب می‌شوند)
\item [ieeetr-fa.bst] این سبک متناظر با \lr{ieeetr.bst} می‌باشد. (مراجع مرتب نمی‌شوند)
\item [plainnat-fa.bst] این سبک متناظر با \lr{plainnat.bst} می‌باشد. نیاز به بستهٔ \lr{natbib} دارد. (مراجع مرتب می‌شوند)
\item [chicago-fa.bst] این سبک متناظر با \lr{chicago.bst} می‌باشد. نیاز به بستهٔ \lr{natbib} دارد. (مراجع مرتب می‌شوند)
\item [asa-fa.bst] این سبک متناظر با \lr{asa.bst} می‌باشد. نیاز به بستهٔ \lr{natbib} دارد. (مراجع مرتب می‌شوند)
\end{description}
\doublespacing

با استفاده از استیلهای فوق می‌توانید به انواع مختلفی از مراجع فارسی و لاتین ارجاع دهید. به عنوان نمونه مرجع 
\cite{Omidali82phdThesis}
 یک نمونه پروژه دکترا (به فارسی) و مرجع 
\cite{Vahedi87} یک نمونه مقاله مجله فارسی است.
مرجع 
\cite{Amintoosi87afzayesh}  یک نمونه  مقاله کنفرانس فارسی و
مرجع 
\cite{Pedram80osool} یک نمونه کتاب فارسی با ذکر مترجمان و ویراستاران فارسی است. مرجع 
\cite{Khalighi07MscThesis} یک نمونه پروژه کارشناسی ارشد انگلیسی و
\cite{Khalighi87xepersian} هم یک نمونه متفرقه  می‌باشند.

مراجع 
\cite{Gonzalez02book,Baker02limits} 
نمونه کتاب و مقاله انگلیسی هستند.
استیل مورد استفاده در این \پ \lr{acm-fa} است که خروجی آنرا در بخش مراجع می‌توانید مشاهده کنید.
نمونه  خروجی سبک \lr{asa-fa} در شکل \ref{fig:asafa} آمده است.

\begin{figure}[t]
\centering
\includegraphics[width=.8\textwidth]{asa-fa-crop.pdf}
\caption{نمونه خروجی با سبک \lr{asa-fa}}
\label{fig:asafa}
\end{figure} 

\subsection{ نحوه استفاده از سبک‌های فارسی}


برای استفاده از بیب‌تک باید مراجع خود را در یک فایل با پسوند \lr{bib} ذخیره نمایید. یک فایل \lr{bib} در واقع یک پایگاه داده از مراجع\LTRfootnote{Bibliography Database}  شماست که هر مرجع در آن به عنوان یک رکورد از این پایگاه داده
با قالبی خاص ذخیره می‌شود. به هر رکورد یک مدخل\LTRfootnote{Entry} گفته می‌شود. یک نمونه مدخل برای معرفی کتاب \lr{Digital Image Processing} در ادامه آمده است:

\singlespacing
\begin{LTR}
\begin{verbatim}
@BOOK{Gonzalez02image,
  AUTHOR =      {Rafael Gonzalez and Richard Woods},
  TITLE =       {Digital Image Processing},
  PUBLISHER =   {Prentice-Hall, Inc.},
  YEAR =        {2006},
  EDITION =     {3rd},
  ADDRESS =     {Upper Saddle River, NJ, USA}
}
\end{verbatim}
\end{LTR}
\doublespacing

در مثال فوق، \lr{@BOOK} مشخصه‌ی شروع یک مدخل مربوط به یک کتاب و \lr{Gonzalez02book} برچسبی است که به این مرجع منتسب شده است.
 این برچسب بایستی یکتا باشد. برای آنکه فرد به راحتی بتواند برچسب مراجع خود را به خاطر بسپارد و حتی‌الامکان برچسب‌ها متفاوت با هم باشند معمولاً از قوانین خاصی به این منظور استفاده می‌شود. یک قانون می‌تواند فامیل نویسنده‌ی اول+دورقم سال نشر+اولین کلمه‌ی عنوان اثر باشد. به \lr{AUTHOR} و $\dots$ و \lr{ADDRESS} فیلدهای این مدخل گفته می‌شود؛ که هر یک با مقادیر مربوط به مرجع مقدار گرفته‌اند. ترتیب فیلدها مهم نیست. 

انواع متنوعی از مدخل‌ها برای اقسام مختلف مراجع همچون کتاب، مقاله‌ی کنفرانس و مقاله‌ی ژورنال وجود دارد که برخی فیلدهای آنها با هم متفاوت است. 
نام فیلدها بیانگر نوع اطلاعات آن می‌باشد. مثالهای ذکر شده در فایل \lr{MyReferences.bib} کمک خوبی به شما خواهد بود. 
%این فایل یک فایل متنی بوده و با ویرایشگرهای معمول همچون \lr{Notepad++} قابل ویرایش می‌باشد. برنامه‌هایی همچون 
%\lr{TeXMaker}
% امکاناتی برای نوشتن این مدخل‌ها دارند و به صورت خودکار فیلدهای مربوطه را در فایل \lr{bib}  شما قرار می‌دهند.  
با استفاده از سبک‌های فارسی آماده شده، محتویات هر فیلد می‌تواند به فارسی نوشته شود، ترتیب مراجع و نحوه‌ی چینش فیلدهای هر مرجع را سبک مورد استفاده  مشخص خواهد کرد.

نکته: بدون اعمال تنظیمات موردنیاز \lr{Bib\TeX} در \lr{TeXWorks}، مراجع فارسی در استیل‌هایی که مراجع را به صورت مرتب شده چاپ می‌کنند، ترتیب کاملاً درستی نخواهند داشت. برای توضیحات بیشتر \cite{persianbib87userguide} را ببینید یا به سایت پارسی‌لاتک مراجعه فرمایید. تنظیمات موردنیاز در \lr{TeXMaker} اصلاح شده اعمال شده‌اند.

\textbf{برای درج مراجع خود لازم نیست نگران موارد فوق باشید. در فایل 
\lr{MyReferences.bib}
 که همراه با این \پ هست، موارد مختلفی درج شده است و کافیست مراجع خود را جایگزین موارد مندرج در آن نمایید.
}

پس از قرار دادن مراجع خود، یک بار \lr{XeLaTeX} را روی سند خود اجرا نمایید، سپس \lr{bibtex} و پس از آن دوبار \lr{XeLaTeX} را. در \lr{TeXMaker} کلید \lr{F11} و در \lr{TeXWorks} هم گزینه‌ی \lr{BibTeX} از منوی \lr{Typeset}، \lr{BibTeX} را روی سند شما اجرا می‌کنند.

برای بسیاری از مقالات لاتین حتی لازم نیست که مدخل مربوط به آنرا خودتان بنویسید. با جستجوی نام مقاله + کلمه \lr{bibtex}  در اینترنت سایتهای بسیاری همچون \lr{ACM} و \lr{ScienceDirect} را خواهید یافت که مدخل \lr{bibtex} مربوط به مقاله شما را دارند و کافیست آنرا به انتهای فایل \lr{MyReferences} اضافه کنید.

از هر یک از سبکهای \lr{Persian-bib} می‌توانید استفاده کنید، البته اگر از سه استیل آخر استفاده می‌کنید و مایلید که مراجع شما شماره بخورند باید بسته \lr{natbib} را با گزینه \lr{numbers} فراخوانی نمایید.
		% پیوست اول: مدیریت مراجع در لاتک
% !TeX root=../main.tex

\chapter{‌جدول، نمودار و الگوریتم در لاتک}
\label{app:latex:more}
%\thispagestyle{empty}

در این بخش نمونه مثالهایی از جدول، شکل، نمودار، الگوریتم و معادلات ریاضی را در لاتک خواهیم دید.
دقت کنید که در پایان‌نامه‌ها و مقالات، باید قاعدهٔ «ارجاع به جلو%
\LTRfootnote{Forward Referencing}»
رعایت شود؛ یعنی ابتدا در متن به شمارهٔ شکل، جدول یا معادله اشاره شود و بعد از آن (زیر آن) خود شکل، جدول یا معادله رسم شود. (توضیحات بیشتر در قسمت
\ref{sec:floatObjs}).

\section{جدول}
دستور اصلی برای رسم جدول در لاتک 
\verb|tabular|
می‌باشد که جدول
\eqref{tab:motionModels}
با استفاده از آن کشیده شده است؛ در
\verb|tabular|
عرض جدول برابر با مجموع عرض ستون‌ها و حداکثر مساوی عرض متن است.
\begin{table}[ht]
\caption{مدلهای تبدیل.}
\label{tab:motionModels}
\centering
\onehalfspacing
\begin{tabular}{|r|c|l|r|}
	\hline نام مدل & درجه آزادی & تبدیل مختصات & توضیح \\ 
	\hline انتقالی & ۲ & $\begin{aligned} x'=x+t_x \\ y'=y+t_y \end{aligned}$  &  انتقال دوبعدی\\ 
	\hline اقلیدسی & ۳ & $\begin{aligned} x'=x\cos\theta - y\sin\theta+t_x \\ y'=x\sin\theta+y\cos\theta+t_y \end{aligned}$  &  انتقالی+دوران \\ 
	\hline 
\end{tabular} 
\end{table}

برای اینکه عرض جدول قابل کنترل باشد، باید از دستورات
\verb|tabularx|،
\verb|tabulary| یا
\verb|tabu|
استفاده کرد که راهنمای آنها در اینترنت وجود دارد.
مثلاً جدول
\ref{tab:motionModelsCont}
با
\verb|tabularx|
رسم شده که عرض جدول در آن ثابت بوده و ستون‌های از نوع
\verb|X|
عرض خالی جدول را پر می‌کنند.
\begin{table}[ht]
	\caption{مدلهای تبدیل دیگر.}
	\label{tab:motionModelsCont}
	\centering
	\onehalfspacing
	\begin{tabularx}{\textwidth}{|r|c|l|X|}
		\hline نام مدل & درجه آزادی & تبدیل مختصات & توضیح \\ 
		\hline مشابهت & ۴ & $\begin{aligned} x'=sx\cos\theta - sy\sin\theta+t_x \\ y'=sx\sin\theta+sy\cos\theta+t_y  \end{aligned}$  & اقلیدسی+تغییرمقیاس \\ 		
		\hline آفین & ۶ & $\begin{aligned} x'=a_{11}x+a_{12}y+t_x \\ y'=a_{21}x+a_{22}y+t_y \end{aligned}$  & مشابهت+اریب‌شدگی \\
		\hline
	\end{tabularx}
\end{table}

\section{معادلات ریاضی و ماتریس‌ها}
تقریباً هر آنچه دانشجویان برای نوشتن فرمول‌های ریاضی لازم دارند، در کتاب 
\lr{mathmode}
آمده است. کافیست در خط فرمان، دستور زیر را وارد کنید:
\begin{latin}
	\texttt{texdoc mathmode}
\end{latin}
متن زیر شامل انواعی از اشیاء ریاضی است که با ملاحظه کدش می‌توانید با دستورات آن آشنا شوید.\\
شناخته‌شده‌ترین روش تخمین ماتریس هوموگرافی الگوریتم تبدیل خطی مستقیم (\lr{DLT\LTRfootnote{Direct Linear Transform}}) است.  فرض کنید چهار زوج نقطهٔ متناظر در دو تصویر در دست هستند،  $\mathbf{x}_i\leftrightarrow\mathbf{x}'_i$   و تبدیل با رابطهٔ
  $\mathbf{x}'_i = H\mathbf{x}_i$
  نشان داده می‌شود که در آن:
\[\mathbf{x}'_i=(x'_i,y'_i,w'_i)^\top  \]
و
\[ H=\left[
\begin{array}{ccc}
h_1 & h_2 & h_3 \\ 
h_4 & h_5 & h_6 \\ 
h_7 & h_8 & h_9
\end{array} 
\right]\]
رابطه زیر را برای الگوریتم  \eqref{alg:DLT} لازم داریم.
\begin{equation}
\label{eq:DLT_Ah}
\left[
\begin{array}{ccc}
	0^\top & -w'_i\mathbf{x}_i^\top & y'_i\mathbf{x}_i^\top \\ 
	w'_i\mathbf{x}_i & 0^\top & -x'_i\mathbf{x}_i^\top \\ 
	- y'_i\mathbf{x}_i^\top & x'_i\mathbf{x}_i^\top & 0^\top
\end{array} 
\right]
\left(
\begin{array}{c}
	\mathbf{h}^1 \\ 
	\mathbf{h}^2 \\ 
	\mathbf{h}^3
\end{array} 
\right)=0
\end{equation}

\section{الگوریتم}

\subsection{الگوریتم ساده با دستورهای فارسی}
با مفروضات فوق، الگوریتم \lr{DLT} به صورت نشان داده شده در الگوریتم \eqref{alg:DLT}  خواهد بود.
\begin{algorithm}[ht]
\onehalfspacing
\caption{الگوریتم \lr{DLT} برای تخمین ماتریس هوموگرافی.} \label{alg:DLT}
\begin{algorithmic}[1]
\REQUIRE $n\geq4$ زوج نقطهٔ متناظر در دو تصویر 
${\mathbf{x}_i\leftrightarrow\mathbf{x}'_i}$،\\
\ENSURE ماتریس هوموگرافی $H$ به نحوی‌که: 
$\mathbf{x}'_i = H \mathbf{x}_i$.
  \STATE برای هر زوج نقطهٔ متناظر
$\mathbf{x}_i\leftrightarrow\mathbf{x}'_i$ 
ماتریس $\mathbf{A}_i$ را با استفاده از رابطهٔ \ref{eq:DLT_Ah} محاسبه کنید.
  \STATE ماتریس‌های ۹ ستونی  $\mathbf{A}_i$ را در قالب یک ماتریس $\mathbf{A}$ ۹ ستونی ترکیب کنید. 
  \STATE تجزیهٔ مقادیر منفرد \lr{(SVD)}  ماتریس $\mathbf{A}$ را بدست آورید. بردار واحد متناظر با کمترین مقدار منفرد جواب $\mathbf{h}$ خواهد بود.
  \STATE  ماتریس هوموگرافی $H$ با تغییر شکل $\mathbf{h}$ حاصل خواهد شد.
\end{algorithmic}
\end{algorithm}

\subsection{الگوریتم پیچیده و تودرتو با دستورهای فارسی}
الگوریتم \ref{alg:simulation-random}، یک الگوریتم ترکیبی و تودرتو است که با کمک دستورهای بستهٔ \lr{algorithmic} نوشته شده است.

\begin{algorithm}[p]
    \onehalfspacing
    \caption{الگوریتم اجرای برنامهٔ شبیه‌سازی}
    \label{alg:simulation-random}
    \begin{algorithmic}[1]
        \REQUIRE زمان $t_{max}$ به عنوان زمان لازم برای انجام شبیه سازی،\\
        \REQUIRE  گراف شبکه برای شبیه سازی،
        \ENSURE جدول تغییرات گراف از لحظهٔ ۰ تا t.
        \FOR {تمام لحظات در بازهٔ ۰ تا $t_{max}$}
            \FOR {تمام پیوند‌ها}
                \STATE محاسبهٔ ضریب و نرخ انتقال پیوند
                \STATE محاسبهٔ کیفیت و نرخ یادگیری
            \ENDFOR
            \FOR {تمام گره‌ها}
                \STATE محاسبهٔ نرخ انتقال گره
                \STATE محاسبهٔ وضعیت جدید
            \ENDFOR
            \IF {تغییرات از مقدار $\delta$ کمتر است}
                \STATE شکستن حلقه
                \COMMENT{این شرط برای پایان قبل از رسیدن به محدودیت زمانی است، اگر تغییرات کمتر از $\delta$ باشد}
            \ELSIF {زمان اجرای برنامه بیش از حد طول کشیده \AND $t>100$}
                \STATE شکستن حلقه
            \ENDIF
        \ENDFOR
        \PRINT {زمان اجرای برنامه}
        \RETURN {ماتریس تغییرات زمانی}
    \end{algorithmic}
\end{algorithm}

\subsection{الگوریتم با دستورهای لاتین}
الگوریتم \ref{alg:RANSAC} یک الگوریتم با دستورهای لاتین است.

\begin{algorithm}[ht]
\onehalfspacing
\caption{الگوریتم \lr{RANSAC} برای تخمین ماتریس هوموگرافی.} \label{alg:RANSAC}
\begin{latin}
\begin{algorithmic}[1]
\REQUIRE $n\geq4$ putative correspondences, number of estimations, $N$, distance threshold $T_{dist}$.\\
\ENSURE Set of inliers and Homography matrix $H$.
\FOR{$k = 1$ to $N$}
  \STATE Randomly choose 4 correspondence,
  \STATE Check whether these points are colinear, if so, redo the above step
  \STATE Compute the homography $H_{curr}$ by DLT algorithm from the 4 points pairs,
  \STATE $\ldots$ % الگوریتم کامل نیست
  \ENDFOR
  \STATE Refinement: re-estimate H from all the inliers using the DLT algorithm.
\end{algorithmic}
\end{latin}
\end{algorithm}

\section{کد}
درج کد به زبان‌های مختلف به سادگی امکان‌پذیر است. برنامه
\ref{code:matlabEx}
یک قطعه کد
\lr{MATLAB}
را نشان می‌دهد.
\begin{figure}[ht]
	\begin{LTR}
        \singlespacing
		\lstinputlisting[language=MATLAB, caption={نمونه کد \lr{MATLAB}}, label={code:matlabEx}]{MatlabExample.m}
        % \doublespacing
	\end{LTR}
\end{figure}

\section{تصویر}
نمونهٔ یک تصویر را در فصل قبل دیدیم. دو تصویر شیر کنار هم را نیز در شکل
\ref{fig:twoLion}
مشاهده می‌کنید.
\begin{figure}[ht]
\centering 
\subfloat[شیر ۱]{ \label{fig:twolion:one}
\includegraphics[width=0.3\textwidth]{lion}}
%\hspace{2mm}
\subfloat[شیر ۲]{ \label{fig:twolion:two}
\includegraphics[width=0.3\textwidth]{lion}}%
\caption{دو شیر}
\label{fig:twoLion} %% label for entire figure
\end{figure}

\section{نمودار}
لاتک بسته‌هایی با قابلیت‌های زیاد برای رسم انواع مختلف نمودارها دارد. مانند بسته‌های \lr{Tikz} و  \lr{PSTricks}. توضیح اینها فراتر از این پیوست کوچک است.%
\footnote{
مثال‌هایی از بکارگیری بسته
\lr{Tikz}
را می‌توانید در
\url{http://www.texample.net/tikz/examples/}
ببینید. توصیه می‌شود دانشجویانی که قصد درج اشکالی مانند گراف را در سند خود دارند، مثالهایی از سایت مذکور را ملاحظه فرمایند.
}
یک نمودار رسم شده با بستهٔ 
\lr{TikZ}
 در شکل 
\ref{fig:parabola}
نشان داده شده است.
\begin{figure}[t]
\centering
\begin{tikzpicture}[scale=2.5]
  \shade[top color=blue,bottom color=gray!50] 
      (0,0) parabola (1.5,2.25) |- (0,0);
  \draw (1.05cm,2pt) node[above] 
      {$\displaystyle\int_0^{3/2} \!\!x^2\mathrm{d}x$};

  \draw[style=help lines] (0,0) grid (3.9,3.9)
       [step=0.25cm]      (1,2) grid +(1,1);

  \draw[->] (-0.2,0) -- (4,0) node[right] {$x$};
  \draw[->] (0,-0.2) -- (0,4) node[above] {$f(x)$};

  \foreach \x/\xtext in {1/1, 1.5/1\frac{1}{2}, 2/2, 3/3}
    \draw[shift={(\x,0)}] (0pt,2pt) -- (0pt,-2pt) node[below] {$\xtext$};

  \foreach \y/\ytext in {1/1, 2/2, 2.25/2\frac{1}{4}, 3/3}
    \draw[shift={(0,\y)}] (2pt,0pt) -- (-2pt,0pt) node[left] {$\ytext$};

  \draw (-.5,.25) parabola bend (0,0) (2,4) node[below right] {$x^2$};
\end{tikzpicture}
\caption{یک نمودار زیبا با ارقام فارسی و قابلیت بزرگ‌نمایی بسیار، بدون از دست دادن کیفیت.}
\label{fig:parabola}
\end{figure}

\section{نحوه قرارگیری اشیای شناور}
\label{sec:floatObjs}
شکل‌ها، جداول و الگوریتم‌ها در لاتک اشیای شناور محسوب می‌شوند؛ یعنی خود لاتک تصمیم می‌گیرد آنها را در کجای صفحه ترسیم کند تا زیباتر باشد. اما می‌توان به لاتک توصیه کرد که آن را در قسمت خاصی از صفحه رسم کند. برای اینکه قاعدهٔ «ارجاع به جلو» رعایت شود باید فقط از پرچم
\verb|[ht]|
استفاده کرد، که می‌گوید اگر جا شد شکل را دقیقاً در همین مکان و در غیراینصورت در بالای صفحه بعد رسم کن.
بنابراین دستورات درج تصویر، جدول و الگوریتم به صورت زیر باید باشند:

\begin{latin}
\begin{verbatim}
	\begin{figure/table/algorithm}[ht]
		...
	\end{figure/table/algorithm}
\end{verbatim}
\end{latin}


%\baselineskip=.75cm
\onehalfspacing
\chapter*{واژه‌نامه فارسی به انگلیسی}\markboth{واژه‌نامه فارسی به انگلیسی}{واژه‌نامه فارسی به انگلیسی}
\addcontentsline{toc}{chapter}{واژه‌نامه فارسی به انگلیسی}
\thispagestyle{empty}

\englishgloss{Probabilistic}{احتمالی}
\englishgloss{Valuation}{ارزیابی}
\englishgloss{Measure}{اندازه }
\englishgloss{Stably}{پایدار}
\englishgloss{Weak Topology}{توپولوژی ضعیف}
\englishgloss{Powerdomain}{دامنه‌توانی}
\englishgloss{Function Space}{فضای تابع}
\englishgloss{Semantic Domain}{دامنه معنایی}
\englishgloss{Program Fragment}{قطعه‌برنامه}
\englishgloss{Dcpo}{مجموعه جزئاً مرتب کامل جهت‌دار}
\englishgloss{Ordered}{مرتب}
\chapter*{واژه‌نامه  انگلیسی به  فارسی}\markboth{واژه‌نامه  انگلیسی به  فارسی}{واژه‌نامه  انگلیسی به  فارسی}
\addcontentsline{toc}{chapter}{واژه‌نامه  انگلیسی به  فارسی}
\thispagestyle{empty}

\persiangloss{مجموعه جزئاً مرتب کامل جهت‌دار}{Dcpo}
\persiangloss{فضای تابع}{Function Space}
\persiangloss{اندازه }{Measure}
\persiangloss{مرتب}{Ordered}
\persiangloss{دامنه‌توانی}{Powerdomain}
\persiangloss{احتمالی}{Probabilistic}
\persiangloss{قطعه‌برنامه}{Program Fragment}
\persiangloss{دامنه معنایی}{Semantic Domain}
\persiangloss{پایدار}{Stably}
\persiangloss{ارزیابی}{Valuation}
\persiangloss{توپولوژی ضعیف}{Weak Topology}

\printindex
% !TeX root=../main.tex
% در این فایل، عنوان پایان‌نامه، مشخصات خود و چکیده پایان‌نامه را به انگلیسی، وارد کنید.

%%%%%%%%%%%%%%%%%%%%%%%%%%%%%%%%%%%%
\latinuniversity{University of Tehran}
\latincollege{College of Engineering}
\latinfaculty{Faculty of Engineering Science}
\latindepartment{Algorithms and Computation}
\latinsubject{Computer Engineering}
\latinfield{Algorithms and Computation}
\latintitle{Writing projects, theses and dissertations using tehran-thesis class}
\firstlatinsupervisor{First Supervisor}
\secondlatinsupervisor{Second Supervisor}
\firstlatinadvisor{First Advisor}
%\secondlatinadvisor{Second Advisor}
\latinname{Sina}
\latinsurname{Momken}
\latinthesisdate{May 2017}
\latinkeywords{Writing Thesis, Template, \LaTeX, \XePersian}
\en-abstract{
This thesis studies on writing projects, theses and dissertations using tehran-thesis class. It ...
}

\label{LastPage}

\end{document}